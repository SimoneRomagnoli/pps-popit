\section{Design di Dettaglio}
In questa sezione viene effettuata un'analisi del design di dettaglio, evidenziando aspetti interni ai componenti del
sistema, senza esplorare l'implementazione effettiva del codice.

\subsection{Design del Modello}
A partire dall'analisi dei requisiti funzionali del sistema, sono stati individuati gli aspetti fondamentali all'interno
del modello di gioco. Nello specifico, sono emerse due macro-categorie:
\begin{itemize}
    \item \textbf{Elementi Statici} - ovvero tutti gli aspetti che trascendono dalla singola partita (gerarchia e logica 
    delle entità).
    \item \textbf{Elementi Dinamici} - ovvero tutte le dinamiche che dipendono strettamente dalla partita in corso.
\end{itemize}

\subsubsection{Elementi Statici}

\subsubsection{Elementi Dinamici}
Inizialmente, è stato progettato il \texttt{Model} in modo che gestisse tutti gli aspetti dinamici del sistema.
Con l'espansione di tali elementi, il \texttt{Model} si è rivelato essere poco scalabile, oltre a doversi occupare di
aspetti logicamente distanti tra loro: quindi, è stato riprogettato per comunicare con degli appositi \textit{manager}
per ciascuno degli aspetti dinamici rilevati. Nello specifico, sono state riscontrate tre logiche fondamentali del
modello:
\begin{itemize}
    \item \textbf{Gestione delle entità in gioco} - ovvero tutte le interazioni con le entità che devono essere
    aggiornate nel corso della partita.
    \item \textbf{Generazione dei palloncini} - ovvero le politiche di \textit{spawn} dei palloncini, tenendo conto dei
    diversi tipi, della quantità, dell'ordine e del \textit{round} corrente.
    \item \textbf{Gestione dei dati di gioco} - ovvero il mantenimento delle statistiche di una partita, quali i punti
    vita, il denaro e la traccia di gioco.
\end{itemize}

Pertanto, il \texttt{Model} funge esclusivamente da \textit{relay} di messaggi, ridirezionandoli ai manager interessati:
tale design è fortemente ispirato al pattern \textbf{Facade} \cite{FacadePattern} e al tipico pattern di delegazione
spesso utilizzato in \textit{Akka}.

\begin{figure}[H]
    \centering
    \includegraphics[width=.8\linewidth]{img/class-managers}
    \caption{Diagramma dei \textit{manager}.}
    \label{fig:managers}
\end{figure}

Nello specifico, il \textit{manager} delle entità è composto da due comportamenti principali:
\begin{itemize}
    \item \textbf{Running} - in cui vengono ricevuti e gestiti i messaggi provenienti da tutto il dominio escluse le
    entità, a partire dal \texttt{TickUpdate}, ossia il messaggio di aggiornamento del \texttt{GameLoop}.
    \item \textbf{Updating} - in cui attende che tutte le entità di gioco siano state aggiornate dai corrispettivi
    attori, per poi notificare il \texttt{GameLoop} e tornare al comportamento precedente.
\end{itemize}
Questa suddivisione è stata pensata per dare priorità ai messaggi in base allo stato di aggiornamento dei singoli attori
rappresentanti le entità: infatti, nel caso in cui il \textit{manager} ricevesse un messaggio mentre si trova nel
comportamento sbagliato, questi lo accoda per poi soddisfare la richiesta appena possibile.

\begin{figure}[H]
    \centering
    \includegraphics[width=\linewidth]{img/state-entities-manager}
    \caption{Diagramma di stato UML del \textit{manager} delle entità.}
    \label{fig:entities-manager}
\end{figure}

Per quanto riguarda il \textit{manager} della generazione dei palloncini, sono stati individuati due
macro-comportamenti:
\begin{itemize}
    \item \textbf{Waiting} - in cui attende un messaggio che comunica l'inizio del nuovo \textit{round}.
    \item \textbf{Spawning} - in cui effettua la generazione dei palloncini presenti all'interno del \textit{round}:
    tale comportamento si può ulteriormente suddividere in:
    \begin{itemize}
        \item \textbf{Spawning Round} - controlla se ci sono altre \textit{streak} di palloncini da generare: in caso
        positivo ne ordina la generazione, altrimenti ritorna in \texttt{Waiting}.
        \item \textbf{Spawning Streak} - genera i palloncini appartenenti ad una \textit{streak}.
        \item \textbf{Paused} - nel caso il gioco venga messo in pausa durante la generazione di palloncini, anche
        quest'ultima dovrebbe essere temporaneamente sospesa.
    \end{itemize}
\end{itemize}

\begin{figure}[H]
    \centering
    \includegraphics[width=.8\linewidth]{img/state-spawn-manager}
    \caption{Diagramma di stato UML del \textit{manager} della generazione dei palloncini.}
    \label{fig:spawn-manager}
\end{figure}

Infine, il \textit{manager} dei dati di gioco è semplicemente composto da un solo comportamento in cui modifica o
inoltra i dati quando richiesto.