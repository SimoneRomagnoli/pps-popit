%! Author = alessandromarcantoni
%! Date = 18/10/21

\section{Design architetturale}\label{sec:architectural-design}
Data la natura del progetto, si è ritenuto opportuno adottare astrazioni ben consolidate per quanto riguarda
l'architettura.

\subsection{Pattern Architetturali}
In particolare, dato che l'applicazione, in quanto gioco, prevede interazioni con l'utente si è deciso
di sfruttare il pattern \textbf{MVC}: uno dei più noti ed utilizzati.

L'adozione di tale architettura ci ha permesso di tracciare un'efficace separazione di responsabilità
tra i vari componenti:
ciò si è rivelato fondamentale, in fase di design e sviluppo del dominio, per concentrare la propria attenzione
sui propri task, senza quindi doversi preoccupare della rappresentazione.
Inoltre, un ulteriore vantaggio che questo pattern garantisce è quello di poter fornire una diversa
implementazione dell'interfaccia grafica senza dover applicare alcuna modifica ad altre porzioni dell'applicazione.

\subsection{Modello ad Attori}
La presenza di un elevato numero di entità all'interno dell'applicazione ha portato alla luce l'esigenza di gestire,
già a livello architetturale, la possibilità di sfruttare a pieno la potenza della cpu rendendo quindi il programma
concorrente. In particolare, per astrarre dalla gestione problematiche tipiche di sistemi concorrenti
(mutua eclusione e corse critiche) si è deciso di adottare un approccio basato su attori. Dal momento che ogni attore
incapsula al suo interno un flusso di controllo basato su \textit{event loop} e che le interazioni avvengono
esclusivamente attraverso scambio di messaggi, non è necessario gestire meccanismi di sincronizzazione.

Inoltre, abbiamo ritenuto il modello ad attori particolarmente adeguato per la realizzazione dell'applicazione
in quanto essi incapsulano un \textbf{comportamento}, il quale ben si adatta a descrivere l'effettivo comportamento
delle entità all'interno del modello. Tale scelta ci ha quindi garantito un livello di astrazione ed un'espressività
che ha positivamente influenzato anche le fasi successive di design ed implementazione.

\subsubsection{Vantaggi del Paradigma ad Attori}
Di seguito si elencano i vantaggi riscontrati a fronte dell'utilizzo del framework \textit{Akka}:
\begin{itemize}
    \item Espressività del codice relativo al comportamento delle entità di gioco.
    \item Migliore incapsulamento dei task di ogni componente.
    \item Maggior modularità del codice.
    \item Evitata la gestione della sincronizzazione mediante meccanismi quali semafori o monitor.
\end{itemize}



